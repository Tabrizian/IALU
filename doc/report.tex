\documentclass{article}[12pt]
\usepackage{xepersian}
\usepackage{setspace}

\renewcommand{\baselinestretch}{1.5} 

\begin{document}
\begin{centering}
به نام خدا\\
ایمان تبریزیان\\
۹۳۳۱۰۳۲\\
\end{centering}


دستورات پشتیبانی شده توسط این واحد منطق و محاسبات:
\begin{enumerate}
\item ۰۰۰ : جمع
\item ۰۰۱ :‌ تفریق
\item ۰۱۰ : ضرب
\item ۰۱۱ : تقسیم
\item ۱۰۰ : شیفت حسابی به سمت چپ
\item ۱۰۱ : شیفت حسابی سمت راست
\item ۱۱۰ : و کردن
\item ۱۱۱ : یا کردن
\end{enumerate}
توضیحات عملکرد هر یک از بخش‌های فوق:\\
\begin{enumerate}
\item جمع\\
مدار این جمع‌کننده با استفاده از یک جمع‌کننده پیش‌بینی کننده رقم نقلی است و هنگامی که ورودی دستور آن ۰ باشد، دو عدد را با هم جمع می‌کند. در واقع این جمع کننده از دو جمع کننده ۴ بیتی تشکیل شده است که با پشت سر هم قرار دادن آن ما می‌توانیم جمع کننده ۸ بیتی خود را بسازیم. هزینه این جمع کننده برابر فلان است.
\item تفریق\\
مدار تفریق کننده با مدار جمع کننده مشترک است و تنها دستور آن ۱ می‌شود
\item ضرب\\
دستور ضرب نیز با استفاده از ضرب آرایه‌ای پیاده‌سازی شده است و با استفاده از نیم‌جمع‌کننده‌ها و تمام‌جمع‌کننده‌ها پیاده سازی شده است که تاخیر آن برابر فلان و مساحت آن برابر بهمان است.
\item تقسیم\\
الگوریتم تقسیم نیز با استفاده از تقسیم به روش بازیابی پیاده‌سازی شده است، در این ماژول از تفریق کننده انتخاب‌گر استفاده شده است که نحوه عملکرد آن به این صورت است که دو عدد ورودی را از هم کم می‌کند و در صورتی‌که حاصل منفی بود خود عدد و در غیر این صورت حاصل تفریق را بر می‌گرداند. با پشت سر هم گذاشتن این عناصر به راحتی می‌توان یک تقسیم کننده‌ای که یک عدد هشت بیتی را در ورودی می‌گیرد و یک عدد ۸ بیتی برمی‌گرداند که ۴ بیت آن را خارج قسمت (‌۴ بیت سمت راست) و ۴ بیت سمت چپ آن را باقی‌مانده در بر می‌گیرد بر می‌گرداند.
\item شیفت حسابی به چپ\\
این بخش بسیار ساده است و تنها عدد گرفته شده به سمت چپ شیفت می‌دهد و جای خالی ایجاد شده را با رقم ۰ پر می‌کند.
\item
\end{enumerate}
\end{document}
